\huge \textbf{Glosario}
\normalsize
\begin{itemize}
	\item \textbf{Activity: es una componente de la aplicación que contiene una pantalla con la que un usuario puede interactuar. Cada actividad tiene una pantalla asociada a una interfaz de usuario. Normalmente la pantalla suele ocupar toda la pantalla.}
	\item \textbf{Fragment: representa un comportamiento o parte de la interfaz de usuario de una Activity. Dentro de una Activity se pueden combinar varios Fragments. Al igual que una Activity, tiene un ciclo de vida propio y recibe sus propios eventos de entrada. Podría definirse como una \enquote{subactividad}.}
	\item \textbf{View: representa el bloque básico para representar cualquier elemento básico de la interfaz de usuario en Android.}
	\item \textbf{ViewGroup: es un tipo especial de View que puede contener Views a su vez. Este tipo de View es utilizada como base de un Layout y otro contenedores complejos dentro de Android.}
	\item \textbf{Layout: define la estructura para el diseño de una interfaz de usuario, como la UI de una actividad o un widget de una aplicación.}
\end{itemize}
