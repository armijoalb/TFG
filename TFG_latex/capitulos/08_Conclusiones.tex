\chapter{Conclusiones}

En este capítulo se describen las conclusiones posteriores a la realización del Trabajo de Fin de Grado, con el fin de revisar los objetivos que fueron establecidos en los distintos apartados durante su realización y para comprobar la satisfacción de cada uno de ellos.\newline

También se introducen algunas líneas de trabajo futuro, en los que se podría extender o mejorar el funcionamiento de este trabajo.\newline

\section[Conclusiones]{Conclusiones}
En el tercer capítulo de este TFG se establecieron una serie de objetivos, cuyos resultados y conclusiones sobre su cumplimiento son los siguientes:
\begin{itemize}
	\item Se ha llevado a cabo una recopilación lo suficientemente completa sobre la historia de los sistemas recomendadores y destancando algunas herramientas recientes en este ámbito.
	\item Se ha elegido y definido un modelo matemático (TOPTW) y uno heurístico (Greedy) funcionales para la generación de itinerarios de rutas de puntos de interés, ajustados a las preferencias del usuario.
	\item Se ha empelado una muestra de datos reales y fiables de puntos de interés y alojamientos (OpenStreetMap) con los cuales se han obtenido resultados satisfactorios al probarlos en el sistema recomendador.
	\item Se ha conseguido crear una aplicación para móvil funcional e intuitiva la es capaz de generar rutas de puntos de interés y representarla en un mapa interactivo.
	\item Se ha documentado el código desarrollado tanto en la parte gráfica de dicha aplicación como en los procesos internos de la misma. También se han documentado el resto de tareas realizadas durante el Trabajo de Fin de Grado. Con todo esto se ha formado esta memoria final.
\end{itemize}

Por tanto, se han logrado satisfacer todos los objetivos propuestos, se puede concluir de que este Trabajo de Fin Grado consiguió terminarse con éxito.\newline

En su realización fue necesario estudiar y aprender a usar OpenStreetMap, Overpass, OSRM, Google Maps API y Google Directions API. Igualmente, se profundizó en otras ya conocidas como Java y el sistema operativo Android.\newline

En este Trabajo de Fin de Grado se ha podido apreciar el potencial de los sistemas de recomendación y el gran impacto positivo que pueden suponer para los turistas y los sectores relacionados con el turismo. Además, construir dicho sistema fue una tarea muy enriquecedora e interesante para el autor; habiéndole dado experiencia y conocimientos muy valiosos.

\section[Trabajos futuros]{Líneas de trabajo futuro}
Para finalizar, en esta sección se muestran una lista de tareas que podrían ser líneas de trabajo futuro para este sistema desarrollado durante el Trabajo de Fin de Grado. Estas consisten en nuevas características y funcionalidades que podrían estudiarse e implementarse para mejorar este producto. Dichas líneas de trabajo son:
\begin{itemize}
	\item \textbf{Integración con transportes urbanos de las ciudades:} para mejorar aún más la experiencia del usuario se podrían indicar medios de transportes urbanos, como autobuses o metro, que el usuario pueda utilizar si quiere para llegar a diferentes puntos de interés de la ruta.
	\item \textbf{Mejorar el algoritmo de búsqueda de rutas:} se podría mejorar las soluciones mostradas por la aplicación utilizando heurísticas mejores que la utilizada en este proyecto, por ejemplo una heurística GRASP. Debido a la falta de tiempo, se decidió no implementarla.
	\item \textbf{Añadir más idiomas:} al estar la aplicación dirigida al turismo, y como gran parte de los turistas que visitan un país son extranjeros, sería oportuno añadir soporte para más idiomas en la aplicación, como por ejemplo Inglés o Chino.
	\item \textbf{Extender las ciudades a visitar:} se podría meter soporte para más ciudades, ya que la aplicación en este momento es un prototipo funcional en la ciudad de Granada. Debido al poco tiempo restante, se decidió no implementar esta característica.
\end{itemize}