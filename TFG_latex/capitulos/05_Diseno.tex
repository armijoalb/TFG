\chapter{Diseño}
En este capítulo se detalla la arquitectura del sistema.
\section[Diagrama de clases]{Diagrama de clases}
En este apartado se mostrarán los diagramas de clases de los elementos más importantes de la aplicación.\newline

La primera imagen que se muestra es el diagrama de clases del Fragment \enquote{TypesFragment} \ref{fig:fragment_diagram}. Dicho diagrama de clases muestra las siguientes clases:
\begin{itemize}
	\item La clase TypesFragment:esta clase es la encargada de comunicar los elementos gráficos que se encuentra dentro de la lista de alojamientos y puntos de interés, también del botón que inicia el algoritmo de búsqueda de la ruta óptima.
	\item La clase TypesRecyclerFragment: es la encargada de manejar la lista de alojamientos y puntos de interés, además devuelve los elementos seleccionados en la lista al iniciar la búsqueda de la ruta. También se encarga de gestionar la interfaz gráfica de la lista.
	\item La clase CityNodesViewHolder: es la encargada de gestionar los elementos gráficos de los puntos de interés o los alojamientos de forma individual.
	\item La clase TypeViewHolder: es la encargada de gestionar los elementos gráficos de los tipos que aparecen en la lista de alojamientos y puntos de interés.
	\item La clase CityNode: clase que contiene toda la información importante sobre un alojamiento o punto de interés que se muestra en la lista.
	\item La clase TypeOfNode: clase que contiene la información sobre un tipo de nodo de interés.
\end{itemize}
\vspace{0.06in}
En la segunda imagen \ref{fig:main_activity_diagram}, se muestra el diagrama de clases de la actividad principal de la aplicación. Las clases que se muestran en el diagrama son las siguientes:
\begin{itemize}
	\item La clase MapsActivity es la clase clase principal y gestiona el mapa que se muestra y se comunica con la clase TypesFragment para obtener los nodos seleccionados. Esta clase contiene vectores que almacenan información sobre los marcadores que aparecen en el mapa. Dicha información se utiliza también para obtener la matriz de distancias. También contiene otros vectores para almacenar la información contenida en la solución, dichos vectores se utilizan para mandar la información de la solución a la actividad ResultActivity.
	\item La clase DownloadFileFromUR: se encarga de descargar y guardar la información que devuelven las peticiones a los servidores sobre alojamientos, puntos de interés, rutas y matriz de tiempos entre puntos de interés. Esta clase hereda de la clase AsyncTask, lo cual le permite ejecutarse en segundo plano sin que afecte al rendimiento de la interfaz de usuario.
	\item La clase jsonProcessor se encarga de procesar la información que se ha guardado en ficheros tras ser descargada. Esta clase utiliza la clase JsonParser para procesar los archivos. Además, también hereda de la clase AsyncTask por lo que se ejecuta en segundo plano.
	\item La clase JsonParser se encarga de procesar archivos JSON y devuelve la información contenida en los archivos en estructuras que la aplicación puede manipular.
	\item La clase SendNodes se encarga de obtener la matrix de distancias entre los puntos de interés selecionados. Para ello utiliza las clases DownloadFileFromURL y jsonProcessor, también se ejecuta en segundo plano debido a que hereda de la clase AsyncTask.
	\item La clase FindSolution se encarga de ejecutar en segundo plano el algoritmo de búsqueda de rutas y de mandar la solución a la siguiente actividad. Para ello hace uso de la clase PathFinder y la clase Solution.
	\item La clase PathFinder es la clase que contiene el algoritmo de búsqueda de rutas. Dicha clase obtiene la solución al problema mediante un algoritmo Greedy y devuelve un objeto de la clase Solution.
	\item La clase Solution es la que clase que contiene una solución al problema. Esta clase cuenta con un vector que almacena los identificadores de los puntos de la solución, así como dos vectores que almacenan las horas de entrada y de salida de cada uno de los puntos de la solución.
\end{itemize}
\vspace{0.06in}
En la tercera imagen \ref{fig:models_diagram}, se muestra el diagrama de clases del paquete models; dicho paquete está formador por clases que se utilizan para modelar diferentes estructuras dentro del proyecto. Las clases que aparecen en el diagrama son las siguientes:
\begin{itemize}
	\item La clase Solution: clase que contiene una solución al problema. Contiene vectores para almacenar identificadores y horarios de entrada y salida de los lugares por los que pasa la solución.
	\item La clase ModelNode: clase genérica que se utiliza para poder mostrar elementos tanto de la clase CityNode y TypeOfNode; ambas clases están explicadas en la descripción del primer diagrama de clases.
	\item La clase SolutionNode: clase que contiene el nombre, el horario de entrada y salida de un nodo de la solución. Esta clase se utiliza para encapsular los nodos de la solución y acceder a los datos a la hora de mostrar la lista de la solución.
\end{itemize}
\vspace{0.06in}
La cuarta imagen \ref{fig:result_activity_diagram} muestra el diagrama de clases de la Activity ResultActivity, dicha Activity se muestra cuando se obtiene una solución. Las clases que se muestran en el diagrama son las siguientes:
\begin{itemize}
	\item La clase ResultActivity. Esta clase se ocupa de leer los datos mandados por la actividad principal y procesarlos para comunicarselos a la clase SolutionFragment.
	\item La clase SimpleFragmentPagerAdapter. Esta clase se ocupa de ver el número de soluciones que se han encontrado y crear un objeto de la clase SolutionFragment para mostrar cada una de ellas.
	\item La clase SolutionFragment. Esta clase se ocupa de mostrar en un mapa los marcadores de la solución y la lista con la información específica de cada uno de los marcadores. Además se ocupa de calcular el camino entre los distintos marcadores de la solución.
\end{itemize}
\vspace{0.06in}
Por último, en la quinta imagen \ref{fig:solution_recycler_diagram} se muestra el diagrama de clases de la clase SolutionFragment. A continuación se describen cada una de las clases que se muestran en el diagrama:
\begin{itemize}
	\item La clase SolutionFragment se ha definido en la imagen anterior. Es la encargada de gestionar todas las vistas que muestran la solución.
	\item La clase FindRoutes. Esta clase se encarga de mandar una petición a un servidor para obtener la ruta óptima entre los nodos de la solución y procesarla, después manda la información procesada a la clase SolutionFragment para que dibuje la ruta.
	\item La clase SolutionRecyclerAdater se encarga de manejar la lista con los nodos de la solución y de mostrarlos en una lista.
	\item La clase SolutionNode. Es una clase que contiene la información sobre un nodo de la solución.
	\item La clase SolutionNodeViewHolder. Es una clase que utiliza la información de un objeto de la clase SolutionNode  y la muestra dentro de la lista que gestiona la clase SolutionRecyclerAdapter.
\end{itemize}
\newgeometry{scale=1}
\thispagestyle{empty}
{%
	\label{fig:fragment_diagram}
	\centering
	\includegraphics[scale=.95,angle=90]{imagenes/fragment_class_diagram.pdf}
	\captionof{figure}{Diagrama de clases del fragment}
	\par
}
\restoregeometry

\newgeometry{scale=1}
\thispagestyle{empty}
{%
	\label{fig:main_activity_diagram}
	\centering
	\includegraphics[scale=.95,angle=90]{imagenes/main_activity_class_diagram.pdf}
	\captionof{figure}{Diagrama de clases de la actividad principal}
	\par
}
\restoregeometry

\newgeometry{scale=1}
\thispagestyle{empty}
{%
	\label{fig:models_diagram}
	\centering
	\includegraphics[scale=.95,angle=90]{imagenes/models_package.pdf}
	\captionof{figure}{Diagrama de clases del paquete models}
	\par
}
\restoregeometry

\newgeometry{scale=1}
\thispagestyle{empty}
{%
	\label{fig:result_activity_diagram}
	\centering
	\includegraphics[scale=.95,angle=90]{imagenes/result_activity.pdf}
	\captionof{figure}{Diagrama de clases del activity ResulActivity}
	\par
}
\restoregeometry

\newgeometry{scale=1}
\thispagestyle{empty}
{%
	\label{fig:solution_recycler_diagram}
	\centering
	\includegraphics[scale=.95,angle=90]{imagenes/result_fragment.pdf}
	\captionof{figure}{Diagrama de clases de la clase SolutionFragment}
	\par
}
\restoregeometry

\section[Interfaz de Usuario]{Interfaz de Usuario}

A continuación se mostrarán todos los elementos que conforman la interfaz de usuario de la aplicación.\newline

Lo primero que se muestra es la actividad principal, esta cuenta con un mapa por el que se puede navegar y una lista de alojamientos y puntos de interés, entre los cuales el usuario puede elegir un alojamiento y el número de puntos de interés que desee \ref{fig:main_activity_map}\ref{fig:main_activity_list}.\newline

Para poder ver la lista, se debe deslizar la pestaña \enquote{Sitios interesantes} hacia arriba, así tendremos acceso a la lista completa, para explorar todos los alojamientos y puntos de interés, debemos hacer scroll hacia arriba sobre la lista. Dicha lista muestra primero los alojamientos y tras estos los puntos de interés, que se organizan en \enquote{Museos}, \enquote{Miradores} y \enquote{Monumentos}.\newline

Dentro la lista, cada uno de los elementos contiene un caja en la que se puede pulsar para seleccionar o deseleccionar de dicho elemento. Dentro de las vistas de los tipos de punto de interés, se encuentra también una caja, dicha caja permite seleccionar o deseleccionar todos los puntos de interés de ese tipo.\newline

Tras elegir el alojamiento los puntos de interés que el usuario desee, se ejecuta el algoritmo y cuando este termina se abre una nueva actividad en la cual se muestran diferentes rutas al problema, por defecto se muestra la primera ruta, para mostrar otras rutas, se debe pulsar sobre los tabs llamados \enquote{"Solución X"} para mostar otras soluciones.\newline

Dentro de cada una de las soluciones, se puede pulsar sobre los marcadores para mostrar información sobre los mismos. Además, en la parte inferior de la pantalla se muestra una lista oculta que muestra la información detalla de la ruta; para poder ver dicha información se debe deslizar hacia arriba en \enquote{Descripción ruta final} o pulsar. Dentro de la lista, se muestra en orden los puntos de interés que contiene la ruta, el primer punto que se muestra es el alojamiento seleccionado; además, cada uno de los puntos de interés muestra la hora aproximada de entrada y de salida de dicho punto de interés. Debajo se muestran imágenes de ejemplo sobre esto \ref{fig:result_activity}\ref{fig:result_activity_2}.


\newpage
\begin{figure}[H]
	\centering
	\includegraphics[width=50mm]{imagenes/main_activity_map}
	\caption{Mapa mostrando los puntos de interés y alojamientos sobre el mapa}
	\label{fig:main_activity_map}
\end{figure}
\vspace{0.06in}
\begin{figure}[H]
	\centering
	\subfigure{\includegraphics[width=50mm]{imagenes/main_activity_list}}
	\subfigure{\includegraphics[width=50mm]{imagenes/main_activity_list_2}}
	\caption{Lista de alojamientos y puntos de interés disponibles para seleccionar}
	\label{fig:main_activity_list}
\end{figure}

\vspace{0.06in}
\begin{figure}
	\centering
	\subfigure{\includegraphics[width=50mm]{imagenes/solution}}
	\subfigure{\includegraphics[width=50mm]{imagenes/list_solution}}
	\caption{Mapa mostrando solución y lista de puntos de interés de la solución}
	\label{fig:result_activity}
\end{figure}

\begin{figure}[H]
	\centering
	\subfigure{\includegraphics[width=50mm]{imagenes/solution_2}}
	\subfigure{\includegraphics[width=50mm]{imagenes/solution_3}}
	\caption{Diferentes soluciones mostradas en un mapa}
	\label{fig:result_activity_2}
\end{figure}
