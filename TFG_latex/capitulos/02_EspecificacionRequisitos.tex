\chapter{Especificación de  Requisitos}
\section[Requisitos]{Requisitos}
En este apartado se resumirán los requisitos funcionales y no funcionales del proyecto, dividiéndolos en los requisitos de la interfaz de usuario y los requisitos internos de la aplicación.
\subsection[Requisitos interfaz]{Requisitos de la interfaz del usuario}
\subsubsection[Requisitos funcionales]{Requisitos funcionales}
\begin{itemize}
	\item Mapa de la ciudad elegida por el usuario.
	\item Formulario de generación de rutas.
	\begin{itemize}
		\item Alojamiento: hostales y hoteles.
		\item Preferencias del usuario: museos, miradores, puntos históricos.
	\end{itemize}
	\item Especificación de las rutas: dibujado de los nodos de las rutas.
	\item Información sobre los marcadores.
\end{itemize}
\subsubsection[Requisitos no funcionales]{Requisitos no funcionales}
\begin{itemize}
	\item Interfaz: la interfaz deberá ser atractiva, ligera y lo más intuitiva posible.
\end{itemize}

\subsection[Requisitos internos]{Requisitos de la aplicación}
\subsubsection[Requisitos funcionales]{Requisitos funcionales}
\begin{itemize}
	\item Recepción de peticiones del cliente.
	\item Cálculo y retorno de rutas óptima.
	\item Publicación de lista de alojamientos.
\end{itemize}
\subsubsection[Requisitos no funcionales]{Requisitos no funcionales}
\begin{itemize}
	\item Envío de peticiones de puntos de interés y alojamientos. (Overpass y OSM)
	\item Importación de información sobre puntos de interés y alojamientos.
	\item Envío de peticiones a servidor de cálculo de matrices de tiempos.
	\item Importación de matriz de tiempos.
\end{itemize}

\section[Casos de uso]{Diagramas de casos de uso}
\subsection[Interfaz de usuario]{Interfaz de usuario}
\subsection[Aplicación]{Aplicación móvil}



\section[Fuente de los datos]{Fuente de los datos}
\subsection[OSM]{Open Street Map}
\subsection[Overpass]{Overpass API}
\subsection[OSRM]{Open Source Routing Machine}

