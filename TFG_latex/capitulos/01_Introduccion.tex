\chapter{Introducción}

\section[Motivación]{Motivación}
Como consecuencia de la globalización, las naciones están cada vez más conectadas. Esto se debe en gran parte a los ordenadores, a Internet; y desde hace unos años a los dispositivos móviles.\newline

Hoy en día casi todo el mundo cuenta con un móvil, el cual utiliza para todo tipo de cosas: redes sociales (Twitter, Facebook, etc...), consultar su cuenta bancaria, escuchar música, ver películas, editar documentos o incluso pagar con él.\newline

Una consecuencia de esto es el aumento en el turismo mundial, en el cual cada año crece más el número de turistas y los ingresos generados por el turismo. Según los últimos datos ofrecidos por el UNWTO \cite{unwto_resumen}, el turismo representa el 10\% del PIB mundial. Según dicha fuente también, el turismo representa el 7\% de las exportaciones mundiales y aporta uno de cada diez puestos de trabajos en todo el mundo. Además, en el año 2016 hubo más de 1200 millones de turistas y se prevé que para 2030 haya 1800 millones de turistas. Este último año ha habido en España unos 82 millones de turistas internacionales, lo que ha generado 87.000 millones de euros, suponiendo un $12.4\%$ más que el año anterior.\newline

Por todo esto, debería aprovecharse el potencial del turismo y adaptarlo a la tecnología actual; mejorando la calidad de las visitas, adaptándose a sus preferencias y necesidades. Con este propósito existen los sistemas de recomendación.\newline

Un sistema de recomendación ofrece al turista encontrar los recursos adecuados a sus preferencias ofreciéndole una relación de puntos de interés filtrados y ordenados.\newline

La propuesta de sistema de recomendación a desarrollar en este proyecto pretende ofrecer itinerarios personalizados que incluyan rutas asociadas a los intereses del usuario y que maximicen la satisfacción de este. El producto será una aplicación móvil desarrollada en Android que ejecutará un algoritmo en Java basado el el problema Tourist Trip Desing Problem. Dicho algoritmo usa una implementación de una heurística voraz (Greedy) y retornará la mejor solución encontrada en una clase contendora, la cual permite dibujar los diferentes puntos de interés de la solución y la ruta asociada a dichos puntos. Para este proyecto se utilizará como ejemplo la ciudad de Granada.
\section[Tourist Trip Design Problem]{Tourist Trip Design Problem}
El \textit{Tourist Trip Design Problem} (TTDP) intenta resolver el problema de obtener una ruta que contenga el número mayor posible de puntos de interés sin violar ninguna restricción que el turista tenga; algunas restricciones pueden ser económicas o de tiempo. Este problema fue mencionado por primera vez por Vansteenwegen y Oudheusden en 2007 \cite{first_article_TTDP}.\newline

Este problema es una extensión del problema llamado \textit{Orienteering Problem} (OP). En el OP, existen localizaciones con un beneficio asociado las cuales deben ser visitadas en una franja de tiempo; el objetivo de este problema es maximizar el beneficio total visitando solamente una vez cada localización. Dicho problema fue introducido en 1984 por Tsiligirides \cite{first_article_OP}.\newline

Con el tiempo han aparecido extensiones del OP, por ejemplo el \textit{Team Orienteering Problem} (TOP), desarrollado en 1996 (Chao et al.) \cite{Chao}. En dicho problema existen varios equipos los cuales tienen que elegir rutas que no pueden contener puntos de interés seleccionados por otros equipos; el objetivo es maximizar la puntuación total sin exceder el tiempo máximo. Un equipo se puede interpretar como un día dentro de un viaje de múltiples días. Una extensión de este mismo problema es el \textit{Team Orienteering Problem with Time Windows} (TOPTW), en el cual un punto de interés puede visitarse  dentro de su ventana de tiempo específica (por ejemplo, el horario de un museo) (Vanteenwegen et al. 2009) \cite{TOPTW}.\newline

Otra extensión del OP es el \textit{Time Dependent Orienteering Problem} (TDOP), en el cual se considera el tiempo necesario para llegar desde un punto de interés a otro punto de interés (Formin and Lingas 2002) \cite{TDOP}. La combinación de los dos problemas anteriores da resultado al \textit{Time Dependent Orienteering Team Problem with Time Windows} (TDTOPTW) (Garcia et al. 2010) \cite{TDTOPTW}. Otro problema es el \textit{Multi Constrained TOPTW}, que considera múltiples restricciones como pueden ser un presupuesta además del tiempo (Sylejmani et al. 2012) \cite{multiconstrained_toptw}.\newline

Algunas extensiones más recientes son el DPTOP, \textit{TOP with Decreasing Profits}, en el cual se utiliza una función para calcular el valor de un punto de interés según el tiempo (Murat Afsar y Labadie 2013) \cite{dptop}; y el \textit{Clustered OP} (Angelelli et al. 2014) \cite{clustered_op}, donde los puntos de interés se asignan en grupos. Cada grupo tiene asignado un valor y este valor forma parte de beneficio total si todos los puntos de interés contenidos en dicho punto son visitados en la ruta.
\section[Objectivos]{Objetivos}
Los objetivos concretos que persigue este proyecto son los siguientes:
\begin{enumerate}
	\item Estudio del estado del arte de los sistemas de planificación de rutas con recomendaciones, y de los sistemas de información sobre puntos de interés en las ciudades.
	\item Definición y diseño de modelos y herramientas de recomendación de itinerarios ajustados a las necesidades y preferencias de los turistas.
	\item Validación de modelos y métodos, empleando pruebas y datos de puntos de interés en ciudades de interés turístico.
	\item Obtención de un prototipo software integrado en una aplicación para dispositivos móviles.
	\item Creación de documentación técnica, entregables y memoria final del proyecto.
\end{enumerate}
	
	

