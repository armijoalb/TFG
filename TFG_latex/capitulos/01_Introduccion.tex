\chapter{Introducción}

\section[Motivación]{Motivación}
Como consecuencia de la globalización, las naciones están cada vez más conectadas. Esto se debe en gran parte a los ordenadores, a Internet; y desde hace unos años a los dispositivos móviles.\newline

Hoy en día casi todo el mundo cuenta con un móvil, el cual utiliza para todo tipo de cosas: redes sociales (Twitter, Facebook, etc...), consultar su cuenta bancaria, escuchar música, ver películas, editar documentos o incluso pagar con él.\newline

Una consecuencia de esto es el aumento en el turismo mundial, en el cual cada año crece más el número de turistas y los ingresos generados por el turismo. Según los últimos datos ofrecidos por el UNWTO \cite{unwto_resumen}, el turismo representa el 10\% del PIB mundial. Según dicha fuente también, el turismo representa el 7\% de las exportaciones mundiales y aporta uno de cada diez puestos de trabajos en todo el mundo. Además, en el año 2016 hubo más de 1200 millones de turistas y se prevé que para 2030 haya 1800 millones de turistas. Este último año ha habido en España unos 82 millones de turistas internacionales, lo que ha generado 87.000 millones de euros, suponiendo un $12.4\%$ más que el año anterior.\newline

Por todo esto, debería aprovecharse el potencial del turismo y adaptarlo a la tecnología actual; mejorando la calidad de las visitas, adaptándose a sus preferencias y necesidades. Con este propósito existen los sistemas de recomendación.\newline

Un sistema de recomendación ofrece al turista encontrar los recursos adecuados a sus preferencias ofreciéndole una relación de puntos de interés filtrados y ordenados.\newline

La propuesta de sistema de recomendación a desarrollar en este proyecto pretende ofrecer itinerarios personalizados que incluyan rutas asociadas a los intereses del usuario y que maximicen la satisfacción de este. El producto será una aplicación móvil desarrollada en Android que ejecutará un algoritmo en Java basado el el problema Tourist Trip Desing Problem. Dicho algoritmo usa una implementación de una heurística voraz (Greedy) y retornará la mejor solución encontrada en una clase contendora, la cual permite dibujar los diferentes puntos de interés de la solución y la ruta asociada a dichos puntos. Para este proyecto se utilizará como ejemplo la ciudad de Granada.
\section[Antecedentes y estado del arte]{Antecedentes y estado del arte}
	
	

