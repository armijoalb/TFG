\chapter*{}
%\thispagestyle{empty}
%\cleardoublepage

%\thispagestyle{empty}



\cleardoublepage
\thispagestyle{empty}

\begin{center}
	{\large\bfseries Software para el diseño de rutas con 
		puntos de interés}\\
\end{center}
\begin{center}
	Alberto Armijo Ruiz\\
\end{center}
\noindent{\textbf{Palabras clave}: Turismo, Rutas, Puntos de Interés, Aplicaciones móviles.}\\

\vspace{0.7cm}
\noindent{\textbf{Resumen}}\\

El objetivo de este trabajo ha sido diseñar y desarrollar de un prototipo de app para el diseño de rutas turísticas con puntos de interés teniendo en cuenta las preferencias del usuario.\newline

A partir de información de diferentes tipos sobre puntos de interés (obtenida mediante la interacción con varios servidores), las preferencias del usuario y un algoritmo de cálculo de rutas; la app obtiene y dibuja en un mapa la ruta recomendada para el usuario.\newline

Para el desarrollo de la app fueron esenciales los conocimientos y la experiencia en programación adquirida a lo largo del grado, en especial la aportada por las asignaturas \enquote*{Metaheurística} y \enquote*{Programación de dispositivos móviles}; las cuales el autor cursó durante el tercer y cuarto curso del Grado en Ingeniería Informática de la Universidad de Granada. Además, fue necesaria la consulta de fuentes bibliográficas para el estudio del problema de diseño de rutas, así como el estudio y uso de herramientas de desarrollo y gestión de proyectos. 

\cleardoublepage


\thispagestyle{empty}


\begin{center}
{\large\bfseries Software for the desing of routes with points of interest}\\
\end{center}
\begin{center}
Alberto Armijo Ruiz\\
\end{center}

%\vspace{0.7cm}
\noindent{\textbf{Keywords}:Tourism, Routes, Points of Interest, Mobile Applications.}\\

\vspace{0.7cm}
\noindent{\textbf{Abstract}}\\
The objective of this work was to design and develop an app's prototipe for the design of turistic routes with points of interest given the user's preferences.\newline

With the information about different types of points of interest (obtained through the interaction with different servers), the user's preferences and an route calculation algorithm; the app gets and draws in a map the recommended route for the user.\newline

For the development of this app were essential the knowledge and experience in programming acquired along the degree, specially those from the subjects \enquote*{Metaheurísticas} and \enquote*{Programación de dispositovs móviles}; which the autor attended during the third and fourth year of the \enquote*{Grado en Ingeniería Informática} of the University of Granada. In addition, it was necessary to consult bibliographic sources for the study of route design problem, as well as the study and use of development and project management tools.

\chapter*{}
\thispagestyle{empty}

\noindent\rule[-1ex]{\textwidth}{2pt}\\[4.5ex]

Yo, \textbf{Alberto Armijo Ruiz}, alumno de la titulación Grado en Ingeniería Informática de la \textbf{Escuela Técnica Superior
de Ingenierías Informática y de Telecomunicación de la Universidad de Granada}, con DNI 26256219V, autorizo la
ubicación de la siguiente copia de mi Trabajo Fin de Grado en la biblioteca del centro para que pueda ser
consultada por las personas que lo deseen.

\vspace{6cm}

\noindent Fdo: Alberto Armijo Ruiz

\vspace{2cm}

\begin{flushright}
Granada a 7 de Junio de 2018 .
\end{flushright}


\chapter*{}
\thispagestyle{empty}

\noindent\rule[-1ex]{\textwidth}{2pt}\\[4.5ex]

D. \textbf{David A. Pelta Mochcovsky}, Profesor del Área de Ciencias de la Computación e Inteligencia Artificial del Departamento DECSAI de la Universidad de Granada.\newline

Dña. \textbf{Marina Torres Anaya}, Investigadora en el Departamento DECSAI de la Universidad de Granada.

\vspace{0.5cm}

\textbf{Informan:}

\vspace{0.5cm}

Que el presente trabajo, titulado \textit{\textbf{Software para el diseño de rutas con puntos de interés}},
ha sido realizado bajo su supervisión por \textbf{Alberto Armijo Ruiz}, y autorizamos la defensa de dicho trabajo ante el tribunal
que corresponda.

\vspace{0.5cm}

Y para que conste, expiden y firman el presente informe en Granada a 7 de Junio de 2018.

\vspace{1cm}

\textbf{Los directores:}

\vspace{5cm}

\noindent \textbf{David A. Pelta Mochcovsky} \\
\noindent \textbf{Marina Torres Anaya}


