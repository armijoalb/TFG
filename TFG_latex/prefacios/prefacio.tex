\chapter*{}
%\thispagestyle{empty}
%\cleardoublepage

%\thispagestyle{empty}



\cleardoublepage
\thispagestyle{empty}

\begin{center}
	{\large\bfseries Software para el diseño de rutas con 
		puntos de interés: Software para el diseño de rutas con 
		puntos de interés en dispositivos móviles}\\
\end{center}
\begin{center}
	Alberto Armijo Ruiz\\
\end{center}
\noindent{\textbf{Palabras clave}: Sistema recomendador, Touring Trip Design Problem, Turismo, Dispositivo móvil.}\\

\vspace{0.7cm}
\noindent{\textbf{Resumen}}\\

El objetivo de este trabajo ha sido documentar los aspectos más significativos del diseño, desarrollo y despliegue de un sistema recomendador de turismo para la generación de rutas de puntos de interés en ciudades que se ejecute en dispositivos móviles.\newline
Para ello, fueron esenciales los conocimientos y la experiencia en programación adquirida a lo largo del grado, en especial la aportada por las asignaturas \enquote*{Metaheurística} y \enquote*{Programación de dispositivos móviles}; las cuales el autor cursó durante el tercer y cuarto curso del Grado en Ingeniería Informática de la Universidad de Granada. \newline
El sistema resultante es una sistema recomendador de turismo que consta de una aplicación y, a través de la interacción con diferentes servidores para obtener la información sobre los puntos de interés y un algoritmo para obtener rutas integrado en la app, obtiene y dibuja diferentes rutas entre las cuales el usuario puede elegir. Además, el usuario puede utilizar filtros para generar diferentes rutas, dependiendo de sus gustos o necesidades.

\cleardoublepage


\thispagestyle{empty}


\begin{center}
{\large\bfseries Software for the desing of routes with points of interest:Software for the desing of routes with points of interest on mobile devices.}\\
\end{center}
\begin{center}
Alberto Armijo Ruiz\\
\end{center}

%\vspace{0.7cm}
\noindent{\textbf{Keywords}: Recommender system, Touring Trip Design Problem, Tourism, Mobile device.}\\

\vspace{0.7cm}
\noindent{\textbf{Abstract}}\\
The objective of this work was to compile the most significant aspects of design, development and deployment of a tourism recommender system for the creation of POI's routes in cities executed in mobile devices.\newline
The knowledge and experience acquired during the degree were essential, especially those from the subjects \enquote*{Metaheurísticas} and \enquote*{Programación de dispositivos móviles}; which the autor attended during the third and fourth year of the \enquote*{Grado en Ingeniería Informática} of the University of Granada.\newline
The resulting system is a tourism system recommender with an app that, through the interaction with differents servers used for getting information about the POI and an algorithm for obtaining routes integrated in the app, gets and draws differents routes among which the user can choose. The user can also use filters to obtain different routes, depending on his needs and his likings.

\chapter*{}
\thispagestyle{empty}

\noindent\rule[-1ex]{\textwidth}{2pt}\\[4.5ex]

Yo, \textbf{Alberto Armijo Ruiz}, alumno de la titulación Grado en Ingeniería Informática de la \textbf{Escuela Técnica Superior
de Ingenierías Informática y de Telecomunicación de la Universidad de Granada}, con DNI 26256219V, autorizo la
ubicación de la siguiente copia de mi Trabajo Fin de Grado en la biblioteca del centro para que pueda ser
consultada por las personas que lo deseen.

\vspace{6cm}

\noindent Fdo: Alberto Armijo Ruiz

\vspace{2cm}

\begin{flushright}
Granada a 24 de Abril de 2018 .
\end{flushright}


\chapter*{}
\thispagestyle{empty}

\noindent\rule[-1ex]{\textwidth}{2pt}\\[4.5ex]

D. \textbf{David Alejandro Pelta}, Profesor del Área de Models of Decision and Optimization del Departamento DECSAI de la Universidad de Granada.

\vspace{0.5cm}

\textbf{Informan:}

\vspace{0.5cm}

Que el presente trabajo, titulado \textit{\textbf{Software para el diseño de rutas con puntos de interés, Software para el diseño de rutas con puntos de interés en dispositivos móviles}},
ha sido realizado bajo su supervisión por \textbf{Alberto Armijo Ruiz}, y autorizamos la defensa de dicho trabajo ante el tribunal
que corresponda.

\vspace{0.5cm}

Y para que conste, expiden y firman el presente informe en Granada a 12 de Mayo de 2018 .

\vspace{1cm}

\textbf{Los directores:}

\vspace{5cm}

\noindent \textbf{David Alejandro Pelta}


